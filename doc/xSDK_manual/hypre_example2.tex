In the next example (Program \ref{Hypre_BelosEx.cpp}), we will examine how to
use hypre preconditioners with Belos solvers.

\begin{lstinputlisting}[caption=Hypre\_BelosEx.cpp,label=Hypre_BelosEx.cpp]{src/Hypre_BelosEx.cpp}
\end{lstinputlisting}

\paragraph{Lines 1--99}
These lines are not substantially different from the previous example.  We
defined convenient typedefs, then set up our operator, solution vector, and
right hand side.

\paragraph{Lines 102--108}
This time, we have elected not to use a hypre linear solver, so we only set the
parameters related to the AMG preconditioner.  Again, it is very important to
set the maximum number of iterations if you wish to use AMG as a preconditioner.

\paragraph{Lines 111--118}
Create the hypre preconditioner.  This time, we specify that we would like to
precondition rather than solve, since we will be using a Belos linear solver.

\paragraph{Lines 121--124}
Create a Belos::LinearProblem that encapsulates the operator, solution vector,
right-hand side, and preconditioner.  We also specify that our operator is
Hermitian so that Belos allows us to use PCG.

\paragraph{Lines 127--135}
Create a Belos linear solver.  The Belos solvers have many parameters, but we
only specify the convergence tolerance and verbosity (what information will be
printed).

\paragraph{Line 138}
Solve the linear system using a Belos pseudo-block conjugate gradient solver
with hypre's BoomerAMG preconditioner.