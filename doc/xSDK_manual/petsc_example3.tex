The following example (Program \ref{Tpetra_KSPEx.cpp}) illustrates this process
in greater detail.  Note that this example does not contain a single of PETSc
code.

\begin{lstinputlisting}[caption=Tpetra\_KSPEx.cpp,label=Tpetra_KSPEx.cpp]{src/Tpetra_KSPEx.cpp}
\end{lstinputlisting}

\paragraph{Lines 1--9}
Include statements

\paragraph{Lines 12--24}
Typedefs and using statements to make the code more readable

\paragraph{Lines 26--29}
Set up MPI

\paragraph{Lines 31--40} 
Parse command line arguments.  This program allows the user to specify the
filename for the matrix, the tolerance for the linear solve, and which linear
solver is used.  A list of all valid linear solver options can be found at
\url{http://www.mcs.anl.gov/petsc/petsc-current/docs/manualpages/KSP/KSPType.html}.

\paragraph{Lines 42--46}
Set up the linear system by reading the matrix from a file, setting the initial
guess for the solution as $\vec{0}$ and setting the right hand side as
$\vec{1}$.

\paragraph{Lines 48--54}
Set up the Ifpack2 Jacobi preconditioner.

\paragraph{Lines 56--59}
Set the maximum number of iterations, convergence tolerance, and which PETSc
KSP solver is being used.

\paragraph{Lines 61--64}
Set up the linear problem for the Belos solver.

\paragraph{Lines 66-68}
Solve the linear system.